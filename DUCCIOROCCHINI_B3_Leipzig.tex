\documentclass{beamer}

\usetheme{Frankfurt}
\usecolortheme{spruce}

\title{Opportunities for Use Cases implementation of WP4 B-Cubed developments}

\author{Duccio Rocchini}

\date{March 26th 2025}

\institute{
\includegraphics[width=.5\textwidth]{"../images/b3logo.png"}
}

\begin{document}

\maketitle

\AtBeginSection[] 
{
\begin{frame}
\frametitle{Outline}
\tableofcontents[currentsection]
\end{frame}
}

\section{Links}

\begin{frame}{Cubes}
\centering
\includegraphics[width=.9\linewidth]{"../images/ebvcube"}
\bigskip
\url{https://github.com/EBVcube}
\end{frame}

\begin{frame}{Main links between WP4 and WP6}

\begin{itemize}
	\item \textbf{WP4}: Task 4.1 - Suitability cube; \textbf{WP6}: Task 6.1 - Case study I - Global and continental biodiversity change
	\pause \item \textbf{WP4}: Task 4.2 - Dissimilarity cube; \textbf{WP6}: Task 6.3: Case study III - Regional indicators in Europe
	\pause \item \textbf{WP4}:  Task 4.3 - Network invasibility cube; \textbf{WP6}: Task 6-2 - Case study II - Biological invasions in South Africa 
\end{itemize}
\end{frame}


\begin{frame}{Task 6.1 - Case study I - Global and continental biodiversity change}
\centering
\includegraphics[width=.4\textwidth]{"../images/npj_communities.jpeg"}

\smallskip
\scriptsize{Task 4.1 - Suitability cube}
\end{frame}


\begin{frame}{Task 6.3: Case study III - Regional indicators in Europe}
\centering 
\includegraphics[width=.8\textwidth]{"../images/anticipating.jpg"}

\smallskip
\scriptsize{Task 4.2 - Dissimilarity cube}

\scriptsize{Task 4.4 - Using B3 for deep learning}
\end{frame}

\begin{frame}{Task 6-2 - Case study II - Biological invasions in South Africa}
\centering
\includegraphics[width=.7\textwidth]{"../images/african_bird_atlas.png"}

\smallskip
\scriptsize{Task 4.3 - Network invasibility cube}
\scriptsize{Task 4.5: Insight in the conditions for reliable national or regional species status and trends from GBIF data}
\end{frame}


\section{Products and accessible science}

\begin{frame}{Products, Guidelines, Minimum data needs}
\textbf{Products}
	\begin{itemize}
		\item Diversity Maps 
		\item Uncertainty maps 
	\end{itemize}
\textbf{Guidelines}
	\begin{itemize}
		\item Free and Open Source code (GitHub)  \includegraphics[width=.05\textwidth]{"../images/gitlogo.png"}
	\end{itemize}
\textbf{Mimum data needs}
	\begin{itemize}
		\item Data at wide spatial scales (Europe, worldwide)
	\end{itemize}
\end{frame}

\begin{frame}{Accessible science}
\includegraphics[width=.9\textwidth]{"../images/mee_2021_2.png"}
\end{frame}

\begin{frame}
\frametitle{The feeling of our maps for colorblind people}
\centering
\includegraphics[width=.9\textwidth]{"../images/simul1.png"}\\
\end{frame}

\begin{frame}
\frametitle{Solutions}
\centering
\includegraphics[width=\textwidth]{"../images/cblindplot_package.png"}\\
\end{frame}

\begin{frame}\frametitle{Info}
\centering
\includegraphics[width=\textwidth]{"../images/environmetrics.png"}\\
\end{frame}

\end{document}
